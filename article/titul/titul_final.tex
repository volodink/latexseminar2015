\documentclass[a4paper,14pt]{article}

\usepackage{extsizes}
\usepackage{cmap}

%%% Работа с русским языком
\usepackage[T2A]{fontenc}
\usepackage[utf8]{inputenc}
\usepackage[english,russian]{babel}   %% загружает пакет многоязыковой вёрстки
\usepackage{indentfirst}

\usepackage[top=15mm,bottom=20mm,left=25mm,right=10mm]{geometry}

\usepackage{setspace} % Интерлиньяж
\onehalfspacing % 1.5 интервал

\begin{document}

\newpage

\thispagestyle{empty}
\begin{center}
МИНОБРНАУКИ РОССИИ

Федеральное государственное бюджетное образовательное учреждение 

высшего профессионального образования

<<Пензенский государственный технологический университет>>
(ПензГТУ)

\vspace{36pt}

Факультет информационных и образовательных технологий

Кафедра <<Информационные технологии и системы>>

Дисциплина <<Языки программирования>>

\vspace{36pt}

К\,У\,Р\,С\,О\,В\,А\,Я Р\,А\,Б\,О\,Т\,А

на тему: <<Разработка программы сопровождения базы данных 

на языке ANSI C>>

(предметная область – <<Склад>>)

\vspace{72pt}

ПОЯСНИТЕЛЬНАЯ ЗАПИСКА

ПензГТУ 3.230400.4.ПЗ

\vspace{72pt}

\parbox{12cm}{
Выполнил: студент гр. 14ИС1ба Иванов И.И.

Проверил: ст. преподаватель каф. ИТС Володин К.И.

Работа защищена с оценкой: \hrulefill
}

\vfill

Пенза 2014
\end{center}

\newpage

\section{Регрессия}

\end{document}
